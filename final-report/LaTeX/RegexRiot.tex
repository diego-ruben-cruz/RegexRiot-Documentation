\documentclass[conference]{IEEEtran}
\usepackage{cite}
\usepackage{amsmath,amssymb,amsfonts}
\usepackage{algorithmic}
\usepackage{url}
\usepackage{graphicx}
\graphicspath{{./images/}}
\usepackage{textcomp}
\usepackage{xcolor}
\usepackage{listings}
\lstdefinelanguage{JavaScript}{
  keywords={const, let, var, function, return, if, else, for, while, do, switch, case, break},
  keywordstyle=\color{blue}\bfseries,
  ndkeywords={class, export, boolean, throw, implements, import, this},
  ndkeywordstyle=\color{darkgray}\bfseries,
  identifierstyle=\color{black},
  sensitive=false,
  comment=[l]{//},
  morecomment=[s]{/*}{*/},
  commentstyle=\color{purple}\ttfamily,
  stringstyle=\color{red}\ttfamily,
  morestring=[b]',
  morestring=[b]"
}
\lstdefinelanguage{Java}{
  keywords={abstract, assert, boolean, break, byte, case, catch, char, class, const, continue, default, do, double, else, enum, extends, final, finally, float, for, goto, if, implements, import, instanceof, int, interface, long, native, new, package, private, protected, public, return, short, static, strictfp, super, switch, synchronized, this, throw, throws, transient, try, void, volatile, while},
  keywordstyle=\color{blue}\bfseries,
  ndkeywords={boolean, byte, char, class, double, enum, float, int, long, short, var},
  ndkeywordstyle=\color{purple}\bfseries,
  identifierstyle=\color{black},
  sensitive=false,
  comment=[l]{//},
  morecomment=[s]{/*}{*/},
  commentstyle=\color{gray}\ttfamily,
  stringstyle=\color{red}\ttfamily,
  morestring=[b]',
  morestring=[b]"
}
\lstset{
  basicstyle=\ttfamily\small,
  breaklines=true,
  columns=fullflexible,
  showstringspaces=false,
  commentstyle=\color{gray}\upshape
}

\def\BibTeX{{\rm B\kern-.05em{\sc i\kern-.025em b}\kern-.08em
    T\kern-.1667em\lower.7ex\hbox{E}\kern-.125emX}}
\begin{document}

\title{RegexRiot \\A Java Library for Building Regular Expressions}

\author{\IEEEauthorblockN{Diego Cruz}
    \IEEEauthorblockA{\textit{Dept. of Computer Science} \\
        \textit{San Jose State University}\\
        San Jose, United States \\
        diego.cruz@sjsu.edu}
    \and
    \IEEEauthorblockN{Rahul Kandekar}
    \IEEEauthorblockA{\textit{Dept. of Computer Science} \\
        \textit{San Jose State University}\\
        San Jose, United States \\
        rahul.kandekar@sjsu.edu}
    \and
    \IEEEauthorblockN{Saptarshi Sengupta}
    \IEEEauthorblockA{\textit{Dept. of Computer Science} \\
        \textit{San Jose State University}\\
        San Jose, United States \\
        saptarshi.sengupta@sjsu.edu}

}


\maketitle

\begin{abstract}
    Regular expressions are essential tools for developers working with text data.
    However, constructing regular expressions can be a tedious and error-prone task when dealing with longform string literals.
    To simplify this process for Java/Kotlin developers, we have developed a Java library called RegexRiot.
    The library offers a straightforward and intuitive way of constructing regular expressions by using a chain of
    function calls that describe the pattern to match.
    While similar projects exist for JavaScript, there is a lack of such libraries for Java/Kotlin.
    This project aims to bridge this gap and provide a solution for developers working with Java/Kotlin.
\end{abstract}

\begin{IEEEkeywords}
    Regular Expressions, Java, JavaScript, Kotlin, Library, Pattern Matching, Parsing
\end{IEEEkeywords}

\section{Introduction}
When creating regex expressions, developers will often have a hard time with keeping track of exactly
what parts of the regex expression accomplish which tasks. For example, keeping track of number ranges,
such as the years 1899 up to 1991, inclusive, can become complicated as the range will include several or-blocks that
will then cause the regex expression to fill the line and possibly go on longer than what fits on the screen,
or cause a long word wrapping which can bloat the regex.
In both cases, if the programmer leaves the code as is, then comes back to it later to modify it for some reason,
there is a problem where the developer will likely have to spend time rereading the regex expression to understand
it again if the documentation was not put into place at the time.

One solution to make regex building and management easier includes the \texttt{magic-regexp} library
which simplifies regex into separate blocks that can break down the expression into separate
groups, which one can define as subgroups.\cite{magic-regexp}
\texttt{magic-regexp} was made as an alternative to the native JavaScript object \texttt{RegExp}.
Figure~\ref{fig:magic-regexp-phonenum-regex} is an example which demonstrates how a regex for a phone number can be made with
\texttt{magic-regexp}. \cite{omereshone2023}

\vfill\eject

\begin{figure}[htbp]
    \centering
    \label{fig:magic-regexp-phonenum-regex}
    \begin{lstlisting}[language=JavaScript]
const NEW_PHONE_RE = createRegExp(
    exactly("+")
        .optionally()
        .and("1")
        .as("country")
        .optionally()
        .and(charIn(" -").optionally())
        .and(charIn("123456789").and(digit.times(3)).as("area"))
        .and(charIn("123456789").and(digit.times(6)).as("rest"))
        .at.lineEnd()
        .at.lineStart()
    );
    \end{lstlisting}
    \caption{\texttt{magic-regexp} Phone Number Regex}
\end{figure}

\section{Design}

The most important features of RegexRiot when conducting research and
designing the library were specified as follows:

\begin{enumerate}
    \item \textbf{Chain of function calls:}\\
          The library will provide a chain of function calls that will allow developers
          to construct regular expressions by describing the pattern they want to match.
          This approach will make it easy for developers to create complex regular expressions
          without the need for extensive knowledge of the underlying syntax.

    \item \textbf{Intuitive syntax:}\\
          The library will use an intuitive syntax that will make it easy for developers
          to understand and use regular expressions. The syntax will be designed to be easy
          to read and will provide a clear description of the pattern to match.

    \item \textbf{Support for Java and Kotlin:}\\
          The library will be developed for Java and Kotlin developers. \cite{oracle_docs1997}
          This will provide a solution for developers working with these languages who are
          looking for an easy-to-use regular expression library.\cite{kotlin_docs2023}

          \vfill\eject

    \item \textbf{Wide range of functionalities:}\\
          The library will provide a wide range of functionalities for pattern matching,
          including capturing groups, non-capturing groups, character classes, quantifiers,
          and more. This will provide developers with a powerful tool for working with text data.
\end{enumerate}

With those aforementioned features taken into account,
some UML diagrams have been provided in the \textbf{Appendices} of this
document.

RegexRiot is heavily inspired by the \texttt{magix-regexp} library, and as such,
follows both the imperative and declarative programming paradigms. \cite{magic-regexp}
RegexRiot follows the imperative programming in that it is based around calling functions
to build individual components of the expression, which can then be used to construct an
overall group that is a larger chunk of the master expression. With respect to the
declarative paradigm, RegexRiot follows in the steps of \texttt{magic-regexp}
in that it provides a simpler and more readable syntax for the creation and
maintenance of regex. It should be no surprise that RegexRiot is primarily a functional programming
library, as it it built upon treating functions as first-class variables, and they can be provided as inputs
for functions which can be ultimately stored as primitive data structures.

\section{Implementation}
The implementation of RegexRiot heavily depends on the usage of tokens to denote specific
primitives such as characters, integers, floats, etc. This was an important coding decision
in the context of the project because it allowed for the simplification of regex expressions
while also reducing the need for multiple comments. With the use of tokens, it is possible to
understand individual blocks of the regex expression, where the programmer would likely only
need to comment on subgroups and the whole overall group to clarify what 'chunks' accomplish
what tasks in parsing.

As shown in Figure~\ref{fig:regexriot-riot-tokens}, RiotTokens is an interface that outlines the tokens that can
be used in conjunction with the RiotString.

\begin{figure}[htbp]
    \centering
    \label{fig:regexriot-riot-tokens}
    \begin{lstlisting}[language=Java]
public interface RiotTokens {
    /**
    * This should be safe to use as argument to use as Argument to RiotSet.chars()
    */
    RiotString DIGIT = newLazyRiot("\\d", true),
                    DOT = newLazyRiot("\\.", true),
                    WORD_CHAR = newLazyRiot("\\w", true),
                    OPEN_BRACE = newLazyRiot("\\(", true),
                    CLOSE_BRACE = newLazyRiot("\\)", true),
                    OPEN_SQ_BRACE = newLazyRiot("\\[", true),
                    CLOSE_SQ_BRACE = newLazyRiot("\\]", true),
                    QUESTION_MARK = newLazyRiot("\\?", true),
                    BOUNDARY = newLazyRiot("\\b", true),
                    SPACE = newLazyRiot("\\s", true),
                    SPACES = oneOrMore(SPACE),
                    PLUS = newLazyRiot("\\+", true);
    /**
    * This is not safe to use as arguments to RiotSet.chars()
    */
    RiotString ANY_CHAR = newLazyRiot("."),
                    LINE_START = newLazyRiot("^", true),
                    Line_END = newLazyRiot("$", true);
    RiotSet HEX = inSetOf(DIGIT)
                    .andChars('a').through('f')
                    .andChars('A').through('F'),
                    HEX_LOWER = inSetOf(DIGIT)
                                    .andChars('a').through('f'),
                    HEX_UPPER = inSetOf(DIGIT)
                                    .andChars('A').through('F');
    int UNLIMITED = -1;

}
            \end{lstlisting}
    \caption{\texttt{RegexRiot} RiotTokens Class}
\end{figure}

A similar approach was used in definining the frequency of a certain group,
where the number of expected instances was simplified from using symbols,
to lexical modifier names. This is shown in Figure~\ref{fig:regexriot-quantifiers}
(note that this is an incomplete class, for the sake of saving space
on the paper).

\begin{figure}[htbp]
    \centering
    \label{fig:regexriot-quantifiers}
    \begin{lstlisting}[language=Java]
public class RiotQuantifiers {
    public static RiotString 
    oneOrNone(RiotString ritex) {
        return ritex.wholeThingOptional();
    }

    public static RiotString 
    zeroOrMore(RiotString ritex) {
        return 
        ritex.wholeZeroOrMoreTimes();
    }

    public static RiotString 
    oneOrMore(RiotString ritex) {
        return 
        ritex.wholeOnceOrMoreTimes();
    }
}
    \end{lstlisting}
    \caption{\texttt{RegexRiot} Incomplete RiotQuantifiers Class}
\end{figure}

For the sake of simplicity, Figure~\ref{fig:regexriot-quantifiers} shows one approach to how
RegexRiot can build expressions, but it is also possible for RiotStrings
to take Strings as input, that may have been previously built by other
RiotStrings, and build upon them to create a larger regex expression.
When defining ranges, RegexRiot has a native way of handling those by way
of defining them as overall 'sets' as shown in Figure~\ref{fig:regexriot-riotset}.

\begin{figure}[htbp]
    \centering
    \label{fig:regexriot-riotset}
    \begin{lstlisting}[language=Java]
public interface RiotSet 
extends RiotString.RiotStringable {
    static <T>RiotSet inSetOf(T seed) {
        return RiotSetImplementations
        .lazyInclusiveSetOf(
            seed.toString());
    }
    static <T>RiotSet outSetOf(T seed) {
        return RiotSetImplementations
        .lazyExclusiveSetOf(
            seed.toString());
    }
    <T>RiotSet and(T extension);
    @FunctionalInterface
    interface RiotSetRange {
        RiotSet through(
            char endCharInclusive);
    }
    RiotSetRange 
    andChars(char startCharInclusive);
    RiotSet complement();
    default RiotString toRiotString() {
        return newLazyRiot(
            toString(), true);
    }
    default RiotString 
    andThen(RiotString expression) {
        return toRiotString()
        .then(expression);
    }
}
    \end{lstlisting}
    \caption{\texttt{RegexRiot} RiotSet Class}
\end{figure}

Those are the core components of the inner workings of RegexRiot,
with additional functions covered to account for taking in a String
as input as opposed to objects defined by RegexRiot.
Further details on how to use the library will be outlined in
the \textbf{Usage} section of the report.

\subsection{Usage}
RegexRiot is a library that is designed to be able to branch out regex into
subgroups and be able to mark them with a certain alias such that they can
be identified upon first reading through the code. Then, additional
comments made can clarify upon larger subgroups as to how they
contribute to the larger overall master expression.

Let us walk through an example of parsing an input for one name,
or another name in the context of a riotString.
Suppose we were searching to match either `Bugs Bunny`
or `DaffyDaffy` in the input. The first thing we would do is
create a RiotString through calling \texttt{riot(String seed)} as shown below:

\begin{lstlisting}[language=Java]
ritex = riot("Bugs")
\end{lstlisting}

But we are not finished yet, so we do not immediately use the \texttt{;}
character. We want to include the \texttt{" Bunny"} portion of the string,
so we shall append \\
\texttt{.then(String extension)} like so:

\begin{lstlisting}[language=Java]
ritex = riot("Bugs").then(" Bunny")
\end{lstlisting}

To assign an alias to \\
\texttt{Bugs Bunny}, we shall use \texttt{.wholeAs(String name)}
to mark it with the tag \texttt{name01} like so:

\begin{lstlisting}[language=Java]
ritex = riot("Bugs")
    .then(" Bunny").wholeAs("name01")
\end{lstlisting}

Now that we have made the first name, we must make use of the
\texttt{.or(RiotString extension)} to begin making the \texttt{DaffyDaffy}
portion of the regex, like so:

\begin{lstlisting}[language=Java]
ritex = riot("Bugs")
    .then(" Bunny").wholeAs("name01")
    .or(riot("Daffy"))
\end{lstlisting}

Note that a new \texttt{riot()} expression was enclosed, which allows a similar
technique to be used as was the case with \texttt{Bugs Bunny}. The main
difference in creating this sub-instance of a RiotString is that we must
use the \texttt{.times(int repeatCount)}6 method to define that \texttt{Daffy} will
occur twice, like so:

\begin{lstlisting}[language=Java]
ritex = riot("Bugs")
    .then(" Bunny").wholeAs("name01")
    .or(riot("Daffy").times(2))
\end{lstlisting}

We will finish this regex by assigning the name \texttt{name02}
to the regex associated with \texttt{DaffyDaffy}.

\begin{lstlisting}[language=Java]
ritex = riot("Bugs")
    .then(" Bunny").wholeAs("name01")
    .or(riot("Daffy").times(2)
        .as("name02"));
\end{lstlisting}

The final expression will appear as shown above, but playing around with
the formatting of the expression as shown below can allow for comments
to surround enclosing blocks of \texttt{.or()} or \texttt{.and()} to allow for
greater clarification when building more complicated regex.

\begin{lstlisting}[language=Java]
answer = "(?<name01>Bugs Bunny)|(?<name02>(Daffy){2})"; // What RegexRiot would generate
ritex = riot("Bugs").
    then(" Bunny").wholeAs("name01")
    .or(riot("Daffy").times(2)
        .as("name02")
    );
\end{lstlisting}

Let us make a more complicated expression where it should match 24-bit
hexadecimal grayscale colors such as \texttt{\#000000}, \texttt{\#a9a9a9},
\texttt{\#848484} as well as 12-bit hexadecimal grayscale colors such as
\texttt{\#FFF} and \texttt{\#000}.

Making the base, we should have the symbol \texttt{\#} appear on ALL instances
of the color, like so:

\begin{lstlisting}[language=Java]
ritex = riot("#")
\end{lstlisting}

From there, we need to make use of \texttt{.and(RiotString extension)},
\texttt{.or(RiotString extension)}, as well as the
\texttt{.times(int repeatCount)} or \texttt{.times(int atLeast, int atMost)}.
The resulting expression will result in the following expression:

\begin{lstlisting}[language=Java]
ritex = riot("#").then(
    DIGIT.or(.inSetOf("").andChars('A').through('F'))
    .or(inSetOf("").andChars('a').through('f'))
).times(1, 2)
\end{lstlisting}

In the above code, \texttt{DIGIT} is a token defined earlier, as explained in
Figure~\ref{fig:regexriot-riot-tokens} of the report, and
\texttt{chars(char inclusiveStartChar).through(char inclusiveEndChar)} is a utility
range method that was also previously defined. However, this does not immediately
account for the fact that following the hexadecimal declaration denoted by
\texttt{\#}, there can be either three characters or six characters, so we can define
the previous code block as a group, and have it possibly repeat twice like so:

\begin{lstlisting}[language=Java]
ritex = riot("#").then(
    DIGIT.or(.inSetOf("").andChars('A').through('F'))
    .or(inSetOf("").andChars('a').through('f'))
).times(1, 2).grouped().then(group(1)).times(2);
resultingRegex = "#((?:\\d|[A-F]|[a-f]){1,2})\\1{2}";
\end{lstlisting}

This provides a basic outline for the functionality of RegexRiot as
it had been designed. What's left is to discuss the types of testing that
had been done, and make a few comments on the current state of the library
as it is initially distributed nearing the end of May 2023.

\subsection{Testing}

Testing was largely conducted with the help of a website known as
\textbf{regextutorials}, which contain both tutorials and
practice exercises on how to make JavaScript Regex, which helps to
understand some of the decisions made for the magic-regexp library
during its implementation. \cite{regextutorials}

In the following examples, modifications would have to be made such
that the double backslash \texttt{\textbackslash\textbackslash} would be reduced to a single backslash
\texttt{\textbackslash} to be an acceptable regex expression in JavaScript when verifying the
output on \textbf{regextutorials}. \cite{regextutorials}

Below is an example of how RegexRiot would generate a regex expression
to match numbers that contain a floating point as per
\textbf{regextutorials}. \cite{regextutorials}

\begin{lstlisting}[language=Java]
answer = "\\d+\\.\\d+"; // What RegexRiot would generate
ritex = oneOrMore(DIGIT).then(DOT).then(oneOrMore(DIGIT));
\end{lstlisting}

The following is an example of how RegexRiot would generate a regex expression
to match the titles of all films produced before 1990 as per the
\textbf{regextutorials} website.\cite{regextutorials}

\begin{lstlisting}[language=Java]
answer = "^.+\\((19[0-8]\\d|\\d{3}|\\d{2}|\\d)\\)"; // What RegexRiot would generate
ritex = LINE_START.then(oneOrMore(ANY_CHAR))
    .then(OPEN_BRACE)
    .then(
        riot("19").then(
                inSetOf("").andChars('0').through('8')
        ).then(DIGIT).or(
                DIGIT.times(3)
        ).or(
                DIGIT.times(2)
        ).or(
                DIGIT
        ).wholeThingGrouped()
    )
    .then(CLOSE_BRACE);
    \end{lstlisting}

Below is an example of how RegexRiot would generate a regex expression to
match the 12 and 24 bit colors whose red/green/blue components are equal to
each other as per the \textbf{regextutorials} website.\cite{regextutorials}

\begin{lstlisting}[language=Java]
answer = "#((?:\\d|[A-F]|[a-f]){1,2})\\1{2}"; // What RegexRiot would generate
ritex = riot("#")
        .then(
                DIGIT.or(
                        inSetOf("").andChars('A').through('F')
                ).or(
                        inSetOf("").andChars('a').through('f')
                )
        ).times(1, 2)
        .grouped()
        .then(group(1))
        .times(2);
\end{lstlisting}

\section{Future Work and Concluding Statements}

RegexRiot has been released as a `.jar` package that is immediately available
on GitHub along with the entirety of the source code which prospective users
can download and build on their local machines. It is clear that this library
remains in infancy, because testing was modeled on a small sample size from
\textbf{regextutorials} where the intention was to
teach regex well enough to know the basics, but not enough to be able to
comprehensively teach users how to cover particular edge cases.
In a few of the exercises, the regex generated by us may contain test escapes
that were not accounted for by \textbf{regextutorials}.\cite{regextutorials}

Future releases may be present in the form of a public Maven repository,
or an entirely different approach can be used and perhaps a package manager
could be made. However, it remains to be seen whether a package manager would
be necessary for a library of such scope.

RegexRiot is an open-source software, and as such, we invite others to use the library
in their programming, and contribute to improve upon it. The documentation for
the software is present in the form of javadoc, but perhaps there is an
opportunity to make a presentable documentation in a \texttt{readme.md}
right from the repository page in GitHub. The \textbf{GitHub Issues} page will provide
an excellent platform for bugfixes and external contributions to the library.

GitHub link to repository:\\
\url{https://github.com/RK22000/RegexRiot}

\bibliographystyle{IEEEtran}
\bibliography{RegexRiot}

\onecolumn
% start of the appendix section, with one column
\appendices
\section*{Appendix A - Incomplete Class Diagram}
\centering
\includegraphics[width=\linewidth]{RegexRiot_Class_Diagram.png}
\label{appendix:classdiagram}

\section*{Appendix B - Object Diagram}
\centering
\includegraphics[width=\linewidth]{RegexRiot_Object_Diagram.png}
\label{appendix:objdiagram}

\end{document}